\section{Correttezza}
\label{correttezza}

Complementari all'implmentaione di un algoritmo sono le prove di correttezza che vengono fornite per sancirne la bontà. Nel caso degli algoritmi di \texttt{Prim, Kruskal e Kruskal-naive}, abbiamo ritenuto opportuno verificare le seguenti ipotesi:

\begin{itemize}
    
    \item che i grafi risultanti dai 3 algoritmi avessero rispettivamente \textbf{stesso peso};
    
    \item che i grafi risultanti dai 3 algoritmi fossero tutti degli \textbf{alberi di supporto}.

\end{itemize}

\subsection{Correttezza Pesi}
\label{correttezza_pesi}
Per provare la correttezza dei pesi calcolati si è pensato di eseguire un semplice controllo di uguaglianza tra i pesi dei grafi MST calcolati dai 3 algoritmi.

\begin{itemize}
    \item \textbf{test\_tot\_pesi(prim, kruskal, kruskal\_naive)}: scorrendo l'array dei grafi risultanti dall'esecuzione degli algoritmi, viene eseguito il confronto tra i pesi, utilizzando l'attributo \texttt{totPeso} di ogni \hyperlink{subsection.2.2}{oggetto grafo}.
\end{itemize}

\subsection{Alberi di supporto}
\label{alberi_supporto}

Per dimostrare la correttezza delle strutture grafi ottenuti si è dimostrato che ogni grafo MST risultante fosse un albero di supporto, per fare ciò, sono state verificate due condizioni, in ogni grafo:

\begin{itemize}
    \item tutti i nodi del grafo sono connessi;
    \item dati n nodi il grafo possiede n-1 archi;
\end{itemize}

A questo proposito sono stati sviluppati tre algoritmi:

\begin{itemize}
    \item \textbf{dfs\_supporto(g, u, n\_nodi, visitati)}: viene eseguita una visita dfs del grafo e riempita una lista per verificare se tutti i nodi vengono visitati;
    
    \item \textbf{test\_albero\_supporto(lista\_grafi)}: viene eseguita dfs\_supporto() per ogni grafo risultante da uno dei 3 algoritmi, per ognuno di essi viene controllato che i nodi visitati siano tutti i nodi del grafo e che, dato n il numero dei nodi, gli archi siano n-1;
    
    \item \textbf{test\_total\_supporto(prim, kruskal\_naive, kruskal)}: viene eseguito test\_albero\_supporto() per i grafi risultanti da tutti e tre gli algoritmi e viene restituito un messaggio a schermo per comunicare il risultato dei test.
    
    
\end{itemize}

\newpage

Di seguito un esempio dell'output di test\_total\_supporto(prim, kruskal\_naive, kruskal):

--------------------

Numero di nodi nel grafo:  10 

Numero di nodi visitati con dfs:  10

Numero di archi nel grafo:  9

É un albero di supporto?:  True

Peso totale dell'albero:  29316

--------------------

Numero di nodi nel grafo:  10

Numero di nodi visitati con dfs:  10

Numero di archi nel grafo:  9

É un albero di supporto?:  True

Peso totale dell'albero:  25217

--------------------

Numero di nodi nel grafo:  10

Numero di nodi visitati con dfs:  10

Numero di archi nel grafo:  9

É un albero di supporto?:  True

Peso totale dell'albero:  16940

--------------------

Numero di nodi nel grafo:  10

Numero di nodi visitati con dfs:  10

Numero di archi nel grafo:  9

É un albero di supporto?:  True

Peso totale dell'albero:  -44448

--------------------
\newline \\
I risultati dei test eseguiti restituiscono tutti esiti positivi.

    

